% To set custom section numbering, use: \setcounter{section}{number-1}
% Example: \setcounter{section}{0} will make the next section "1"

\section{Domain Analysis}
\subsection{Objective of the Laboratory Work}

% TODO: Describe the objective of the laboratory work.
% What is the main goal? What concepts or technologies will be explored?

\subsection{Problem Definition}

% TODO: Define the problem to be solved.
% What are the specific requirements? List any commands, behaviors, or constraints.

\subsection{Used Technologies}

% TODO: Add a subsubsection for each technology used (protocols, interfaces, frameworks).
% For technology diagrams (e.g., UART frame, I2C timing), ask the user to supply an image
% and insert an \IfFileExists figure block below.

% Example:
% \subsubsection{Technology Name}
% Description of the technology, its purpose, and how it relates to the lab work.
%
% \begin{figure}[H]
% \centering
% \IfFileExists{resources/DomainAnalysis/UsedTechnologies/image.png}{%
%     \includegraphics[width=0.7\textwidth]{resources/DomainAnalysis/UsedTechnologies/image.png}%
% }{%
%     \fbox{\parbox{0.68\textwidth}{\centering\color{gray}\small
%         [Image pending --- save to resources/DomainAnalysis/UsedTechnologies/image.png]}}%
% }
% \caption{Description of the figure}
% \label{fig:technology_figure}
% \end{figure}

\subsection{Hardware Components}

% TODO: Fill in subsubsection text for each component below.
% For each peripheral introduced in this lab, copy the example block at the bottom,
% replace component_name/caption, and ask the user to supply the photo.
% Check previous labs' resources/DomainAnalysis/HardwareComponents/ for reusable images
% before requesting new ones. Copy any reusable images locally — never use ../labX.Y/... paths.

% ---- MCU / Development Board ------------------------------------------------
\subsubsection{Arduino Mega 2560}

% TODO: Describe the board — ATmega2560 MCU, 16 MHz clock, 256 KB Flash, 8 KB SRAM, GPIO pins.

\begin{figure}[H]
\centering
\IfFileExists{resources/DomainAnalysis/HardwareComponents/arduino_mega_2560.png}{%
    \includegraphics[width=0.65\textwidth]{resources/DomainAnalysis/HardwareComponents/arduino_mega_2560.png}%
}{%
    \fbox{\parbox{0.63\textwidth}{\centering\color{gray}\small
        [Photo pending --- save to resources/DomainAnalysis/HardwareComponents/arduino\_mega\_2560.png]}}%
}
\caption{Arduino Mega 2560 development board}
\label{fig:arduino_mega_2560}
\end{figure}

% ---- Peripheral Components --------------------------------------------------
% TODO: Add a \subsubsection + figure block for each peripheral used in this lab.
% Copy and adapt the example below for each one:
%
% \subsubsection{Component Name}
% Description of the component: what it is, specifications, role in this lab.
%
% \begin{figure}[H]
% \centering
% \IfFileExists{resources/DomainAnalysis/HardwareComponents/component_name.png}{%
%     \includegraphics[width=0.5\textwidth]{resources/DomainAnalysis/HardwareComponents/component_name.png}%
% }{%
%     \fbox{\parbox{0.48\textwidth}{\centering\color{gray}\small
%         [Photo pending --- save to resources/DomainAnalysis/HardwareComponents/component\_name.png]}}%
% }
% \caption{Component Name}
% \label{fig:component_name}
% \end{figure}

\subsection{Software Components}

% TODO: Fill in the subsubsection text for PlatformIO and Wokwi below.
% If the same screenshots were already used in a previous lab, copy them into
% resources/DomainAnalysis/SoftwareComponents/ (never use ../labX.Y/... paths).

% ---- PlatformIO IDE ---------------------------------------------------------
\subsubsection{PlatformIO}

% TODO: Describe PlatformIO — VS Code extension, build system, environment management.

\begin{figure}[H]
\centering
\IfFileExists{resources/DomainAnalysis/SoftwareComponents/platformio.png}{%
    \includegraphics[width=0.75\textwidth]{resources/DomainAnalysis/SoftwareComponents/platformio.png}%
}{%
    \fbox{\parbox{0.73\textwidth}{\centering\color{gray}\small
        [Screenshot pending --- save to resources/DomainAnalysis/SoftwareComponents/platformio.png]}}%
}
\caption{PlatformIO IDE in VS Code}
\label{fig:platformio}
\end{figure}

% ---- Wokwi Simulator --------------------------------------------------------
\subsubsection{Wokwi}

% TODO: Describe Wokwi — browser/VS Code embedded simulator, supported boards.

\begin{figure}[H]
\centering
\IfFileExists{resources/DomainAnalysis/SoftwareComponents/wokwi.png}{%
    \includegraphics[width=0.75\textwidth]{resources/DomainAnalysis/SoftwareComponents/wokwi.png}%
}{%
    \fbox{\parbox{0.73\textwidth}{\centering\color{gray}\small
        [Screenshot pending --- save to resources/DomainAnalysis/SoftwareComponents/wokwi.png]}}%
}
\caption{Wokwi simulator running the lab circuit}
\label{fig:wokwi}
\end{figure}

\subsection{System Architecture and Justification}

% TODO: Describe the system's layered architecture and justify design choices.
% List the layers (APP, SRV, ECAL, MCAL, MCU, ECU) and explain each one's role.

\subsection{Case Study - Relevant Applicability of the Solution}

% TODO: Describe real-world applications of the technologies and patterns used.
% Provide examples from industry, education, or other domains.
