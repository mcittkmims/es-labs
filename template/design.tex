
\section{Design}

\subsection{System Architecture Diagrams}

% TODO: Describe the system architecture.
% Present both high-level component structure and detailed layered architecture.

\subsubsection{Component-Level Architecture}

% TODO: Describe the general system architecture with a structural diagram.
% Remember: Use width and height constraints on all figures.
% Example: \includegraphics[width=0.85\textwidth, height=0.5\textheight, keepaspectratio]{...}
%
% \begin{figure}[H]
% \centering
% \includegraphics[width=0.85\textwidth, height=0.5\textheight, keepaspectratio]{resources/Design/SystemArchitectureDiagrams/system_structural_diagram.png}
% \caption{System structural diagram of the application}
% \label{fig:system_structural_diagram}
% \end{figure}
%
% Describe each component and the hardware-software interface.

\subsubsection{Layered System Architecture}

% TODO: Describe the layered architecture diagram and each layer's responsibilities.
%
% \begin{figure}[H]
% \centering
% \includegraphics[width=0.75\textwidth, height=0.5\textheight, keepaspectratio]{resources/Design/SystemArchitectureDiagrams/layered_architecture.png}
% \caption{Layered architecture of the system}
% \label{fig:layered_architecture}
% \end{figure}
%
% Describe each layer and its role in the system.

\subsection{Block Diagrams}

% TODO: Present flowcharts and block diagrams of the algorithm and application logic.
% Use separate per-function flowcharts when modules have multiple complex functions.
%
% \begin{figure}[H]
% \centering
% \includegraphics[width=0.65\textwidth, height=0.5\textheight, keepaspectratio]{resources/Design/BlockDiagrams/app_flowchart.png}
% \caption{Flowchart of the application logic}
% \label{fig:app_flowchart}
% \end{figure}
%
% Describe the algorithm step by step.

\subsection{Electrical Schematics}

% TODO: Present the circuit schematic and describe the hardware configuration.
%
% \begin{figure}[H]
% \centering
% \includegraphics[width=0.65\textwidth, height=0.5\textheight, keepaspectratio]{resources/Design/ElectricalSchematics/electric_schema.png}
% \caption{Circuit schematic}
% \label{fig:circuit_schematic}
% \end{figure}

\subsubsection{Component Specification}

% TODO: List and describe circuit components with their specifications.

\subsubsection{Circuit Connections}

% TODO: Describe the electrical connections between components.

\subsubsection{Hardware Configuration}

% TODO: Present the Wokwi configuration or other hardware setup details.
%
% \begin{lstlisting}[language=json, caption=Wokwi hardware configuration file, label=lst:wokwi_config]
% { ... }
% \end{lstlisting}

\subsection{Project structure}

% TODO: Document the project directory structure.
%
% \begin{lstlisting}[caption=Project directory structure, label=lst:project_structure]
% Labs/
% |-- platformio.ini
% |-- ...
% \end{lstlisting}

\subsection{Modular Implementation}

% TODO: Describe each software module's implementation following the layered architecture.
% Present the interface (.h file) and implementation (.cpp file) for each module.
% Include per-function flowcharts and operational logic for each significant function.
% Use the following structure for each layer:
%
% 1. Brief description of the layer's role
% 2. Interface header: show the public API and available methods
% 3. Implementation highlights: explain critical code sections
% 4. Per-function flowcharts: one diagram per significant function (not one per layer)
%
% Example subsubsection for a module:
% \subsubsection{MCAL Layer: LED Driver}
%
% Brief description of the module's purpose.
%
% \begin{lstlisting}[language=C++, caption=Module interface (header), label=lst:module_h]
% // Code here
% \end{lstlisting}
%
% Description of implementation.
%
% \begin{lstlisting}[language=C++, caption=Module implementation, label=lst:module_cpp]
% // Code here
% \end{lstlisting}
%
% Per-function flowchart example (0.2\textwidth for narrow flowcharts, 0.15\textwidth for small ones):
%
% \begin{figure}[H]
% \centering
% \includegraphics[width=0.2\textwidth, height=0.5\textheight, keepaspectratio]{resources/Design/ModularImplementation/flowchart_init.png}
% \caption{Flowchart: Function initialization}
% \label{fig:flowchart_init}
% \end{figure}
% \label{fig:module_flowchart}
% \end{figure}
