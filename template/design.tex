
\section{Design}

\subsection{System Architecture Diagrams}

% TODO: Describe the system architecture.
% Present both high-level component structure and detailed layered architecture.

\subsubsection{Component-Level Architecture}

% TODO: Describe the general system architecture with a structural diagram.
% Draw using TikZ with the 'positioning' and 'fit' libraries (already loaded).
% Use explicit at(x,y) coordinates for reliable placement.
% Example component block diagram:
%
% \begin{figure}[H]
% \centering
% \begin{tikzpicture}[
%     comp/.style={draw, rectangle, rounded corners, minimum width=3cm,
%                  minimum height=0.9cm, text centered, align=center, fill=blue!10}]
%     \node[comp] (mcu)    at (0,    0)   {Arduino Mega 2560};
%     \node[comp] (lcd)    at (4.5,  1.8) {LCD 1602 (I2C)};
%     \node[comp] (keypad) at (-4.5, 1.8) {4x4 Keypad};
%     \node[comp] (led)    at (4.5, -1.8) {LEDs};
%     \draw[->] (mcu) -- node[above, font=\tiny]{I2C} (lcd);
%     \draw[<-] (mcu) -- node[above, font=\tiny]{GPIO} (keypad);
%     \draw[->] (mcu) -- node[right, font=\tiny]{GPIO} (led);
% \end{tikzpicture}
% \caption{System structural diagram of the application}
% \label{fig:system_structural_diagram}
% \end{figure}
%
% Describe each component and the hardware-software interface.

\subsubsection{Layered System Architecture}

% TODO: Describe the layered architecture diagram and each layer's responsibilities.
% Draw using TikZ. Example layered architecture:
%
% \begin{figure}[H]
% \centering
% \begin{tikzpicture}[
%     box/.style={draw, rectangle, minimum width=6cm, minimum height=0.8cm,
%                 text centered, fill=blue!10},
%     node distance=0.5cm]
%     \node[box] (app)  {Application Layer (APP)};
%     \node[box] (srv)  [below=of app]  {Service Layer (SRV)};
%     \node[box] (ecal) [below=of srv]  {MCU Abstraction (ECAL)};
%     \node[box] (mcal) [below=of ecal] {Driver Layer (MCAL)};
%     \node[box] (hw)   [below=of mcal] {Hardware (HW)};
%     \foreach \a/\b in {app/srv, srv/ecal, ecal/mcal, mcal/hw}
%         \draw[->] (\a) -- (\b);
% \end{tikzpicture}
% \caption{Layered architecture of the system}
% \label{fig:layered_architecture}
% \end{figure}
%
% Describe each layer and its role in the system.

\subsection{Block Diagrams}

% TODO: Present flowcharts and block diagrams using TikZ (already loaded in settings.tex).
% Use separate per-function flowcharts when modules have multiple complex functions.
%
% COMPLEXITY LIMIT: maximum 8 decision nodes per flowchart.
% If a function branches across more paths, split into:
%   1. A high-level overview flowchart showing state/operation groups
%   2. One sub-diagram per group
% For FSM state diagrams: maximum 8 states per tikzpicture. Split larger FSMs.
%
% Standard flowchart node styles (use \tikzset, NOT deprecated \tikzstyle):
%
% \tikzset{
%     startstop/.style={rectangle, rounded corners, minimum width=2.5cm,
%         minimum height=0.7cm, text centered, align=center, draw=black, fill=gray!15},
%     process/.style={rectangle, minimum width=2.5cm,
%         minimum height=0.7cm, text centered, align=center, draw=black, fill=blue!10},
%     decision/.style={diamond, aspect=2.5, minimum width=2cm,
%         minimum height=0.7cm, text centered, align=center, draw=black, fill=orange!15},
%     arrow/.style={thick,->,>=Stealth}
% }
%
% IMPORTANT: Always use explicit at(x,y) coordinates — never use 'node distance'
% with 'below of=' / 'right of=' for flowcharts. Relative positioning causes overlaps.
% Minimum vertical gaps: 1.8cm process-process, 2.5cm near diamonds, 2.0cm near I/O.
%
% \begin{figure}[H]
% \centering
% \begin{tikzpicture}
%     \node (start) [startstop] at (0,   0.0)  {Start};
%     \node (proc1) [process]   at (0,  -1.8)  {Process 1};
%     \node (dec1)  [decision]  at (0,  -4.3)  {Condition?};
%     \node (proc2) [process]   at (0,  -6.8)  {Process 2};
%     \node (stop)  [startstop] at (0,  -8.6)  {End};
%     \draw [arrow] (start) -- (proc1);
%     \draw [arrow] (proc1) -- (dec1);
%     \draw [arrow] (dec1) -- node[anchor=east]{Yes} (proc2);
%     \draw [arrow] (proc2) -- (stop);
% \end{tikzpicture}
% \caption{Flowchart of the application logic}
% \label{fig:app_flowchart}
% \end{figure}
%
% For FSM state diagrams use the automata library:
%
% \begin{figure}[H]
% \centering
% \begin{tikzpicture}[>=Stealth, auto, node distance=2.8cm,
%         every state/.style={draw, circle, minimum size=1.2cm}]
%     \node[state, initial]   (s0) {IDLE};
%     \node[state] (s1)  [right of=s0] {MENU};
%     \node[state, accepting] (s2) [below of=s1] {RESULT};
%     \path[->]
%         (s0) edge              node {*}    (s1)
%         (s1) edge              node {\#}   (s2)
%         (s2) edge [bend right] node {auto} (s0);
% \end{tikzpicture}
% \caption{FSM state diagram}
% \label{fig:fsm_state_diagram}
% \end{figure}
%
% Describe the algorithm step by step.

\subsection{Electrical Schematics}

% TODO: Draw the circuit schematic using CircuiTikZ (already loaded in settings.tex).
% Use \draw chains with component keywords: R (resistor), C (capacitor), leD* (LED),
% battery1 (battery), short (wire), ground, vcc, and more.
% Full component list: https://tikz.dev/library-circuits
%
% \begin{figure}[H]
% \centering
% \begin{circuitikz}
%     % LED with current-limiting resistor
%     \draw (0,0) node[anchor=east]{Pin 7}
%         to[R, l=220\,\si{\ohm}] (2,0)
%         to[leD*, color=red]     (4,0)
%         node[ground] {};
%     % I2C SDA connection
%     \draw (0,-1.5) node[anchor=east]{SDA (Pin 20)}
%         to[short] (4,-1.5) node[anchor=west]{LCD SDA};
% \end{circuitikz}
% \caption{Circuit schematic}
% \label{fig:circuit_schematic}
% \end{figure}

\subsubsection{Component Specification}

% TODO: List and describe circuit components with their specifications.

\subsubsection{Circuit Connections}

% TODO: Describe the electrical connections between components.

\subsubsection{Hardware Configuration}

% TODO: Present the Wokwi configuration or other hardware setup details.
%
% \begin{lstlisting}[language=json, caption=Wokwi hardware configuration file, label=lst:wokwi_config]
% { ... }
% \end{lstlisting}

% SCREENSHOT REQUIRED (manual action): Capture the running Wokwi simulation circuit.
% - Open the lab's wokwi.toml in VS Code and start the Wokwi simulation.
% - Take a screenshot showing the full circuit with all components visible and the
%   simulation running (LEDs/LCD active, not just the static diagram).
% - Save as: resources/Design/ElectricalSchematics/wokwi_circuit.png
\begin{figure}[H]
\centering
\IfFileExists{resources/Design/ElectricalSchematics/wokwi_circuit.png}{%
    \includegraphics[width=0.85\textwidth]{resources/Design/ElectricalSchematics/wokwi_circuit.png}%
}{%
    \fbox{\parbox{0.83\textwidth}{\centering\color{gray}\small
        [Screenshot pending --- run the Wokwi simulation and save to\\
        resources/Design/ElectricalSchematics/wokwi\_circuit.png]}}%
}
\caption{Wokwi simulation circuit (active run)}
\label{fig:wokwi_circuit}
\end{figure}

\subsection{Project structure}

% TODO: Document the project directory structure.
%
% \begin{lstlisting}[caption=Project directory structure, label=lst:project_structure]
% Labs/
% |-- platformio.ini
% |-- ...
% \end{lstlisting}

\subsection{Modular Implementation}

% TODO: Describe each software module's implementation following the layered architecture.
% Present the interface (.h file) and implementation (.cpp file) for each module.
% Include per-function flowcharts and operational logic for each significant function.
% Use the following structure for each layer:
%
% 1. Brief description of the layer's role
% 2. Interface header: show the public API and available methods
% 3. Implementation highlights: explain critical code sections
% 4. Per-function flowcharts: one diagram per significant function (not one per layer)
%
% Example subsubsection for a module:
% \subsubsection{MCAL Layer: LED Driver}
%
% Brief description of the module's purpose.
%
% \begin{lstlisting}[language=C++, caption=Module interface (header), label=lst:module_h]
% // Code here
% \end{lstlisting}
%
% Description of implementation.
%
% \begin{lstlisting}[language=C++, caption=Module implementation, label=lst:module_cpp]
% // Code here
% \end{lstlisting}
%
% Per-function flowchart — draw directly in TikZ (never use PNG exports):
%
% \tikzset{
%     startstop/.style={rectangle, rounded corners, minimum width=2.5cm,
%         minimum height=0.7cm, text centered, align=center, draw=black, fill=gray!15},
%     process/.style={rectangle, minimum width=2.5cm,
%         minimum height=0.7cm, text centered, align=center, draw=black, fill=blue!10},
%     decision/.style={diamond, aspect=2.5, minimum width=2cm,
%         minimum height=0.7cm, text centered, align=center, draw=black, fill=orange!15},
%     arrow/.style={thick,->,>=Stealth}
% }
%
% \begin{figure}[H]
% \centering
% \begin{tikzpicture}
%     \node (start) [startstop] at (0,  0.0)  {Start};
%     \node (proc1) [process]   at (0, -1.8)  {Initialize};
%     \node (stop)  [startstop] at (0, -3.6)  {Return};
%     \draw [arrow] (start) -- (proc1);
%     \draw [arrow] (proc1) -- (stop);
% \end{tikzpicture}
% \caption{Flowchart: Function initialization}
% \label{fig:flowchart_init}
% \end{figure}
