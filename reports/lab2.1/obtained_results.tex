\section{Obtained Results}

\subsection{Build Process}

The project was compiled using PlatformIO with the \texttt{lab2\_1} environment targeting the Arduino Mega 2560 (ATmega2560). The build completed successfully with no errors or warnings, using only 2.7\,\% of the available Flash memory (6\,768 bytes out of 253\,952 bytes) and 9.8\,\% of SRAM (800 bytes out of 8\,192 bytes).

% Save screenshot to: resources/ObtainedResults/build_results.png
\begin{figure}[H]
\centering
\IfFileExists{resources/ObtainedResults/build_results.png}{%
    \includegraphics[width=0.85\textwidth]{resources/ObtainedResults/build_results.png}%
}{%
    \fbox{\parbox{0.83\textwidth}{\centering\color{gray}\small
        [Screenshot pending --- save to resources/ObtainedResults/build\_results.png]}}%
}
\caption{Successful PlatformIO build output for Lab 2.1 (\texttt{lab2\_1} environment)}
\label{fig:build_results}
\end{figure}

The expected build summary is:

\begin{lstlisting}[caption=PlatformIO build summary, label=lst:build_summary]
RAM:   [=         ]   9.8% (used 800 bytes from 8192 bytes)
Flash: [          ]   2.7% (used 6768 bytes from 253952 bytes)
========================= [SUCCESS] =========================
Environment    Status    Duration
-------------  --------  ------------
lab2_1         SUCCESS   00:00:02.xxx
\end{lstlisting}

\subsection{Simulation Startup}

Upon starting the Wokwi simulation with the \texttt{lab2\_1} firmware, the board boots and the scheduler is initialised. Task 3 fires after an initial 2\,s offset and prints the startup banner via STDIO. The serial terminal shows the welcome message:

\begin{lstlisting}[caption=Expected startup banner in the Wokwi serial terminal, label=lst:startup_output]
========================================
  Lab 2.1 -- Button Press Monitor
  Non-Preemptive Task Scheduler Demo
  Tasks: 3 | Tick base: 10 ms
========================================
GREEN  LED  = short press (< 500 ms)
RED    LED  = long press  (>= 500 ms)
YELLOW LED  = activity blink
Report interval: 10 seconds
========================================
\end{lstlisting}

% Save screenshot to: resources/ObtainedResults/simulation_start.png
\begin{figure}[H]
\centering
\IfFileExists{resources/ObtainedResults/simulation_start.png}{%
    \includegraphics[width=0.75\textwidth]{resources/ObtainedResults/simulation_start.png}%
}{%
    \fbox{\parbox{0.73\textwidth}{\centering\color{gray}\small
        [Screenshot pending --- save to resources/ObtainedResults/simulation\_start.png]}}%
}
\caption{Wokwi simulation start --- startup banner in serial terminal, all LEDs off}
\label{fig:simulation_start}
\end{figure}

\subsection{Button Press Detection}

\subsubsection{Short Press Test ($<$\,500\,ms)}

Pressing the button briefly (less than 500\,ms) causes Task 1 to light the green LED for 1\,500\,ms and set the \texttt{g\_newPress} flag. Task 2 picks up the event on its next 50\,ms tick and drives a 5-blink yellow LED sequence (10 half-cycles at 100\,ms each = 1\,000\,ms total). The green LED and yellow blink are simultaneously visible in the simulation.

% Save screenshot to: resources/ObtainedResults/test_short_press.png
\begin{figure}[H]
\centering
\IfFileExists{resources/ObtainedResults/test_short_press.png}{%
    \includegraphics[width=0.75\textwidth]{resources/ObtainedResults/test_short_press.png}%
}{%
    \fbox{\parbox{0.73\textwidth}{\centering\color{gray}\small
        [Screenshot pending --- save to resources/ObtainedResults/test\_short\_press.png]}}%
}
\caption{Short press result --- green LED lit, yellow LED blinking (5 blinks)}
\label{fig:test_short_press}
\end{figure}

\subsubsection{Long Press Test ($\geq$\,500\,ms)}

Holding the button for more than 500\,ms causes Task 1 to light the red LED and signal a long press. Task 2 starts a 10-blink yellow LED sequence (20 half-cycles = 2\,000\,ms). The red LED and the longer blink sequence are visible simultaneously.

% Save screenshot to: resources/ObtainedResults/test_long_press.png
\begin{figure}[H]
\centering
\IfFileExists{resources/ObtainedResults/test_long_press.png}{%
    \includegraphics[width=0.75\textwidth]{resources/ObtainedResults/test_long_press.png}%
}{%
    \fbox{\parbox{0.73\textwidth}{\centering\color{gray}\small
        [Screenshot pending --- save to resources/ObtainedResults/test\_long\_press.png]}}%
}
\caption{Long press result --- red LED lit, yellow LED blinking (10 blinks)}
\label{fig:test_long_press}
\end{figure}

\subsection{Periodic STDIO Report}

Every 10 seconds, Task 3 prints the statistics for the elapsed window. After performing a mix of short and long presses, the serial terminal shows output similar to the following:

\begin{lstlisting}[caption=Example periodic report after 4 presses (2 short + 2 long), label=lst:periodic_report]
===== [10s Report] =====
Total presses    : 4
Short presses    : 2  (< 500 ms)
Long presses     : 2  (>= 500 ms)
Average duration : 712 ms
========================
\end{lstlisting}

After printing, all counters are reset to zero. A window with no button activity prints all zeroes for counts and average duration.

% Save screenshot to: resources/ObtainedResults/test_report.png
\begin{figure}[H]
\centering
\IfFileExists{resources/ObtainedResults/test_report.png}{%
    \includegraphics[width=0.75\textwidth]{resources/ObtainedResults/test_report.png}%
}{%
    \fbox{\parbox{0.73\textwidth}{\centering\color{gray}\small
        [Screenshot pending --- save to resources/ObtainedResults/test\_report.png]}}%
}
\caption{Periodic STDIO report from Task 3 after a sequence of button presses}
\label{fig:test_report}
\end{figure}
